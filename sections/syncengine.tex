Attraverso il SyncManager è possibile utilizzare tutte le funzionalità del SyncEngine tramite una facile interfaccia, la quale consente al programmatore di sfruttare la creazione dinamica ed automatica degli store e il salvataggio dei dati sul server attraverso dei metodi propri di tale gestore.

Il SyncEngine è stato progettato e realizzato come pacchetto indipendente ed utilizzabile in ogni applicazione che necessita di salvataggio online di dati presenti nella memoria del dispositivo mobile dove tale applicazione viene utilizzata.
Per poter utilizzare tale pacchetto in un qualsiasi progetto progetto sono possibili 2 diverse strade che utilizzano strumenti diversi.

\paragraph*{Cmd 3.1.2}
I framework Sencha (Touch e Ext JS) consentono di creare un \textit{workspace} unico nel quale creare e sviluppare i progetti mobili e desktop; tale workspace contiene:
	\begin{itemize}
	\item le versioni dei framework installati nel sistema;
	\item i progetti creati;
	\item i \textit{package} creati dall'utente o dalla comunità che possono essere utilizzati come librerie esterne in qualsiasi progetto;
	\end{itemize}
Appare chiaro come il SyncEngine si presti ad essere considerato un package all'interno del workspace Sencha; affinché ciò sia possibile è necessario compilare tale libreria tramite il Sencha Cmd in modo che lo stesso lo riconosca e lo renda utilizzabile nei diversi contesti in cui si deciderà di usarlo.

Per creare un workspace ed un package seguire la procedura:
\begin{enumerate}
\item \texttt{sencha -sdk /path/to/sdk generate workspace /path/to/workspace}: tale comando consente di generare un workspace nel percorso specificato utilizzando il framework desiderato;
\item \texttt{sencha -sdk [touch,ext] generate package -type code packageName}: tale comando consente di generare il package \textit{packageName} il cui tipo indica che si tratta di una libreria di codice (le alternative sono \textit{theme}, che identifica un tema grafico, e \textit{locale} che identifica una lingua per uso specifico) che utilizza il framework specificato;
\item \texttt{sencha package build}: dopo aver scritto il codice del pacchetto all'interno della cartella \texttt{src} ed aver modificato il file di configurazione \texttt{package.json}, lanciare tale comando per eseguire la build del package.
\end{enumerate}

In questo modo il pacchetto viene reso disponibile all'interno dello spazio di lavoro a patto di essere richiesto nel file di configurazione, \texttt{app.json}, del progetto che lo deve utilizzare.

\lstset{language=Java}
\begin{lstlisting}[caption={Inclusione di un pacchetto in un'applicazione - Sencha Touch 2}]
{
    "name": "MyApp",
    "requires": [
        "packageName"
    ]
}
\end{lstlisting}

\paragraph*{Apache Cordova}
Con Cordova è molto più semplice usare delle librerie esterne, o comunque dei file di supporto al codice di un'applicazione, in quanto esso è stato creato per essere utilizzato in concomitanza con qualsiasi codice Javascript e non per uno specifico framework.

Per utilizzare un pacchetto esterno è infatti sufficiente inserire la cartella contente i file richiesti all'interno della cartella del progetto e utilizzarli come previsto dal framework usato per lo sviluppo.
Nel caso si Sencha Touch 2 e del SyncEngine è sufficiente utilizzare il nome completo di ogni classe nel dichiarazione apposita nel file \texttt{app.js}:

\begin{lstlisting}[caption={Inclusione di un pacchetto in un'applicazione - Apache Cordova}]
Ext.Loader.setPath({
    'Ext': 'touch/src',
    'org': 'org'
});

Ext.application({
    requires: [
        'org.s2.syncEngine.SyncManager'
    ],
    ...
});
\end{lstlisting}

\paragraph*{Soluzione utilizzata}
Nel caso del SyncEngine è stato possibile utilizzare Sencha Cmd per costruire il pacchetto attraverso dei workaround\footnote{\url{http://www.sencha.com/forum/showthread.php?261175-Add-third-party-libraries},\\\url{http://tinyurl.com/lkwaj6o}} presenti in alcuni forum in quanto, seguendo i manuali Sencha, non era possibile giungere al completamento dell'operazione.
Una volta creato il pacchetto è stato però impossibile utilizzarlo in un'applicazione in quanto Sencha Cmd non ha mai completato con successo un processo di build: in seguito a questo fatto è stato aperto un thread nel forum Sencha\footnote{\url{http://tinyurl.com/lgnme4w}} attraverso il  quale, inizialmente, è stata ricevuta, ma dopo alcuni post la conversazione ha avuto termine senza giungere ad una soluzione.

Si è deciso, quindi, di proseguire con il secondo metodo, utilizzando cioè Cordova per effettuare la build del progetto.
Con tale metodo non è stato riscontrato alcun problema riguardo all'integrazione del SyncEngine.
